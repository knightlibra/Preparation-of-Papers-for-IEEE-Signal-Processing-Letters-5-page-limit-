\documentclass[journal]{IEEEtran}

\ifCLASSINFOpdf
\else
   \usepackage[dvips]{graphicx}
\fi
\usepackage{url}

\hyphenation{op-tical net-works semi-conduc-tor}

\usepackage{graphicx}


\begin{document}

\title{FAST: Feature Aggregation for Detecting Salient Object in Real-Time }

\author{First A. Author, \IEEEmembership{Fellow, IEEE}, Second B. Author, and Third C. Author, Jr., \IEEEmembership{Member, IEEE}}
%\thanks{This paragraph of the first footnote will contain the date on which you submitted your paper for review. It will also contain support information, including sponsor and financial support acknowledgment. For example, ``This work was supported in part by the U.S. Department of Commerce under Grant BS123456.'' }
%\thanks{The next few paragraphs should contain the authors' current affiliations, including current address and e-mail. For example, F. A. Author is with the National Institute of Standards and Technology, Boulder, CO 80305 USA (e-mail: author@boulder.nist.gov).}
%\thanks{S. B. Author, Jr., was with Rice University, Houston, TX 77005 USA. He is now with the Department of Physics, Colorado State University, Fort Collins, CO 80523 USA (e-mail: author@lamar.colostate.edu).}}

\markboth{Journal of \LaTeX\ Class Files, Vol. 14, No. 8, August 2015}
{Shell \MakeLowercase{\textit{et al.}}: Bare Demo of IEEEtran.cls for IEEE Journals}
\maketitle

\begin{abstract}
These instructions give you guidelines for preparing papers for IEEE Signal Processing Letters. Use this document as a template if you are using \LaTeX. Otherwise, use this document as an instruction set. The electronic file of your paper will be formatted further at IEEE. Paper titles should be written in uppercase and lowercase letters, not all uppercase. Do not write ``(Invited)'' in the title. Full names of authors are preferred in the author field, but are not required. Put a space between authors’ initials. The abstract must be a concise yet comprehensive reflection of what is in your article. In particular, the abstract must be self-contained, without abbreviations, footnotes, or references. It should be a microcosm of the full article. The abstract is typically between 100--175 words. The abstract must be written as one paragraph, and should not contain displayed mathematical equations or tabular material. The abstract should include three or four different keywords or phrases, as this will help readers to find it. It is important to avoid over-repetition of such phrases as this can result in a page being rejected by search engines. Ensure that your abstract reads well and is grammatically correct.
\end{abstract}

\begin{IEEEkeywords}
Feature aggregation, real-time, salient object detection, edge information.
\end{IEEEkeywords}


\IEEEpeerreviewmaketitle



\section{Introduction}

\IEEEPARstart{T}{he} goal of salient object detection (SOD) is to identify the most visually distinctive objects in an image. As a fundamental task, a good SOD method should be both fast and accurate. Then it can be applied in a broad range of vision and image processing areas, such as visual tracking \cite{DBLP:journals/pami/MahadevanV13,DBLP:conf/icml/HongYKH15}, image retrieval \cite{DBLP:conf/cvpr/HeFLCLCC12}, and person re-identification \cite{DBLP:conf/cvpr/ZhaoOW13}. Although numerous valuable models have been proposed, it is still difficult for traditional methods to locate salient object accurately especially in some complicated scenarios because of lacking high-level semantic information. 

Recently, with the capability of efficiently extracting high-level semantic information, convolutional neural networks (CNNs) have successfully broken the limits of traditional handcrafted features, and greatly promote the development of SOD models. These CNN based methods\cite{DBLP:conf/iccv/ZhangWLWY17,DBLP:conf/cvpr/LeeTK16} mainly focus on utilizing high-level features extracted from last convolutional layers, which are better at positioning the exact places of salient objects. However, at the same time, they ignore low-level visual information such as spatial coherence, leading to suboptimal results with coarse and blurry object boundaries.
Subsequent methods \cite{DBLP:conf/cvpr/LiuH16,DBLP:conf/iccv/ZhangWLWR17,DBLP:journals/tip/ZhangLLS19,DBLP:journals/pami/HouCHBTT19} solve the aforementioned problems by introducing U-NET \cite{DBLP:conf/miccai/RonnebergerFB15} into SOD to construct enriched feature maps, progressively refining the high-level features with the local information in a recursive way \cite{DBLP:conf/eccv/WangWLZR16,DBLP:conf/cvpr/ZhangWQLW18}, adding extra structure information like the boundary term \cite{DBLP:journals/spl/ZhugeYZL18,DBLP:journals/spl/GuanWQZL19,DBLP:conf/cvpr/LiuHCFJ19}, or using post-processing (CRF) \cite{DBLP:conf/cvpr/LiY16,DBLP:journals/pami/HouCHBTT19} to preserve the object boundaries. While these methods have achieved excellent SOD accuracy, the main inconvenience with these approaches is their low inference speed, which drag them on wide-ranging applications.

In order to achieve real-time SOD, Liu {\em et al.} \cite{DBLP:conf/cvpr/LiuHCFJ19} perform new spatial pyramid pooling module to make computation efficient. However, implementing spatial pyramid module on heavy backbone network is still time-consuming. Li {\em et al.} \cite{DBLP:conf/icassp/LiSTH19} use a lightweight backbone network to fast extract high-level semantic information and a parallel branch with shallow network for structural information as supplementary. Nevertheless, the mutual independence between branches limits the model learning ability. Instead of ‘borrowing’ information from other branches, feature reuse is a more economical thought. This motivates us to find a lightweight model to incorporate multi-level context into encoded features.

In this letter, we propose a novel cross-level feature aggregation strategy, which can enhance the model learning capacity and increase the receptive field simultaneously. 
First, we replicate the lightweight backbone \cite{DBLP:conf/cvpr/Chollet17} and reuse high-level features extracted from the previous backbone to the input of the next backbone to bridge gap between semantic information and structure details. We call this process network-level aggregation. 
Second, we combine features of different phases in the processing path including 'coarse' part and 'fine' part of the network architecture to enhance feature representation ability. This process is called phase-level aggregation.
Then, to better preserve object boundaries, we leverage the salient edge features as complement of the salient object features to accurately locate objects. After these modules, a slight decoder composed of convolution and bilinear upsampling operations is adopted to combine the outputs of each level. To alleviate the precision loss in high-dimension features when the whole network structure depth grows, we use multi-level salient supervision in network training which can boost the gradient backpropagation flow. As shown in Fig. 1, compared with other methods, our novel architecture (FAST) achieves a better balance between speed and accuracy which can generate impressive SOD results with a far faster speed (186 fps).


\section{THE PROPOSED METHOD}
The structure of the proposed method is illustrated in Fig. 2. The novelty of our work mainly manifests in the following several aspects: 
First, to achieve real-time SOD, we propose a novel cross-level feature aggregation strategy to reuse features of lightweight backbone network in both network-level and phase-level. Second, to better preserve object boundaries, we explore the complementary between salient object information and edge information within a lightweight architecture. Through the proposed  novel architecture, extensive experiments show that our method achieves the state-of-the-art performance both in speed and accuracy. In this section, we elaborate on the effectiveness of the key components.
\subsection{Cross-level Feature Aggregation}
We focus on making the fusion of different depth features in networks. Our aggregation strategy is composed of network-level aggregate and phase-level aggregate methods.
\subsubsection{Network-level Aggregation:} The goal of network-level aggregation is reusing high-level features to bridge gap between semantic information and structure details. We stack lightweight backbones by feeding the output of the previous backbone to the next to implement network-level aggregation architecture.  From another perspective, network-level aggregation can be seen as a feature refinement process. We define one backbone process as 
$y=\Psi(x)$. Then we feed the output of encoder $\Psi_{n}$ into next encoder $\Psi_{n+1}$, so the whole network-level aggregation can be formulated as: $Y=\Psi_{n}(\Psi_{n-1}(\dots\Psi_{1}(X)))$.
Network-level aggregation allows these high-level features to be processed again to further evaluate and reassess higher order spatial relationships.
\subsubsection{Phase-level Aggregation:}
As the depth of network grows, only reusing features in a single path could cause  spatial details losing. So the phase-level aggregation is proposed to combine features through encoding period to enhance feature representation ability.  
We make the fusion of different phases in the same depth of backbones. In detail, the output of a certain phase in the previous backbone is contributed to the input of the next backbone in the corresponding phase position. We define a phase process as $\psi_{n}^{i}$ in a backbone $\Psi_{n}(x)$. The corresponding phase of previous backbone network is then defined as $\psi_{n-1}^{i}$. $i$ is the index of the phase and $n$ is the index of backbone network.
Phase-level aggregation can be formulated as:
$x_{n}^{i}=cat(x_{n-1}^{i},x_{n}^{i-1})+\psi(cat(x_{n-1}^{i},x_{n}^{i-1}))$.
Which means the input of $i$th phase in $n$th backbone is given by fusing the $i$th phase output in $(n-1)$th network with the $(i-1)$th phase output in $n$th network. 
Since all backbones have the similar structure, $x_{n-1}^{i}$ has the same  resolution as its corresponding phase $x_{n}^{i-1}$. So phase-level feature fusion
can be produced by concatenation operation and the the $i$th phase learns a residual representation of $cat(x_{n-1}^{i},x_{n}^{i-1})$ to generate the multi-phase context. 

we propose stage-level method (Figure 2(d)) to deliver low-level features and spatial information to semantic understanding.

Our sub-stage aggregation is proposed to combine features through encoding period. reuse high-level features extracted from the previous backbone to the input of the next backbone to bridge gap between semantic information and structure details.
pro- posed Deep Feature Aggregation Network aims to exploit features combined from both network-level and stage-level.


The IEEEtran\_HOWTO.pdf is the complete guide of \LaTeX\ for manuscript preparation included with this stuff. 


\subsection{Information for Authors}

{\em IEEE Signal Processing Letters} allows only four-page articles. A fifth page is allowed for ``References'' only, though ``References'' may begin before the fifth page. Author biographies or photographs are not allowed in Signal Processing Letters. Please review the Information for Authors at for {\em IEEE Signal Processing Letters:} https://signalprocessingsociety.org/publications-resources/ieee-signal-processing-letters/information-authors-spl



\section{Guidelines for Graphics Preparation and Submission}
\label{sec:guidelines}

\subsection{Types of Graphics}
The following list outlines the different types of graphics published in 
{\it IEEE Signal Processing Letters}. They are categorized based on their construction, and use of 
color/shades of gray:

\subsubsection{Color/Grayscale figures}
{Figures that are meant to appear in color, or shades of black/gray. Such 
figures may include photographs, illustrations, multicolor graphs, and 
flowcharts.}

\subsubsection{Line Art figures}
{Figures that are composed of only black lines and shapes. These figures 
should have no shades or half-tones of gray, only black and white.}

\subsubsection{Tables}
{Data charts which are typically black and white, but sometimes include 
color.}



\subsection{Multipart figures}
Figures compiled of more than one sub-figure presented side-by-side, or 
stacked. If a multipart figure is made up of multiple figure
types (one part is lineart, and another is grayscale or color) the figure 
should meet the stricter guidelines.

\subsection{File Formats For Graphics}\label{formats}
Format and save your graphics using a suitable graphics processing program 
that will allow you to create the images as PostScript (PS), Encapsulated 
PostScript (.EPS), Tagged Image File Format (.TIFF), Portable Document 
Format (.PDF), Portable Network Graphics (.PNG), or Metapost (.MPS), sizes them, and adjusts 
the resolution settings. When 
submitting your final paper, your graphics should all be submitted 
individually in one of these formats along with the manuscript.

\subsection{Sizing of Graphics}
Most charts, graphs, and tables are one column wide (3.5 inches/88 
millimeters/21 picas) or page wide (7.16 inches/181 millimeters/43 
picas). The maximum depth a graphic can be is 8.5 inches (216 millimeters/54
picas). When choosing the depth of a graphic, please allow space for a 
caption. Figures can be sized between column and page widths if the author 
chooses, however it is recommended that figures are not sized less than 
column width unless when necessary. 

\begin{figure}
\centerline{\includegraphics[width=\columnwidth]{fig1.png}}
\caption{Magnetization as a function of applied field. Note that ``Fig.'' is abbreviated. There is a period after the figure number, followed by two spaces. It is good practice to explain the significance of the figure in the caption.}
\end{figure}

\begin{table}
\caption{Units for Magnetic Properties}
\label{table}
\small
\setlength{\tabcolsep}{3pt}
\begin{tabular}{|p{25pt}|p{75pt}|p{110pt}|}
\hline
Symbol& 
Quantity& 
Conversion from Gaussian and \par CGS EMU to SI$^{\mathrm{a}}$ \\
\hline
$\Phi $& 
Magnetic flux& 
1 Mx $\to  10^{-8}$ Wb $= 10^{-8}$ V $\cdot$ s \\
$B$& 
Magnetic flux density, \par magnetic induction& 
1 G $\to  10^{-4}$ T $= 10^{-4}$ Wb/m$^{2}$ \\
$H$& 
Magnetic field strength& 
1 Oe $\to  10^{-3}/(4\pi )$ A/m \\
$m$& 
Magnetic moment& 
1 erg/G $=$ 1 emu \par $\to 10^{-3}$ A $\cdot$ m$^{2} = 10^{-3}$ J/T \\
$M$& 
Magnetization& 
1 erg/(G $\cdot$ cm$^{3}) =$ 1 emu/cm$^{3}$ \par $\to 10^{-3}$ A/m \\
4$\pi M$& 
Magnetization& 
1 G $\to  10^{-3}/(4\pi )$ A/m \\
$\sigma $& 
Specific magnetization& 
1 erg/(G $\cdot$ g) $=$ 1 emu/g $\to $ 1 A $\cdot$ m$^{2}$/kg \\
$j$& 
Magnetic dipole \par moment& 
1 erg/G $=$ 1 emu \par $\to 4\pi \times  10^{-10}$ Wb $\cdot$ m \\
$J$& 
Magnetic polarization& 
1 erg/(G $\cdot$ cm$^{3}) =$ 1 emu/cm$^{3}$ \par $\to 4\pi \times  10^{-4}$ T \\
$\chi , \kappa $& 
Susceptibility& 
1 $\to  4\pi $ \\
$\chi_{\rho }$& 
Mass susceptibility& 
1 cm$^{3}$/g $\to  4\pi \times  10^{-3}$ m$^{3}$/kg \\
$\mu $& 
Permeability& 
1 $\to  4\pi \times  10^{-7}$ H/m \par $= 4\pi \times  10^{-7}$ Wb/(A $\cdot$ m) \\
$\mu_{r}$& 
Relative permeability& 
$\mu \to \mu_{r}$ \\
$w, W$& 
Energy density& 
1 erg/cm$^{3} \to  10^{-1}$ J/m$^{3}$ \\
$N, D$& 
Demagnetizing factor& 
1 $\to  1/(4\pi )$ \\
\hline
\multicolumn{3}{p{251pt}}{Vertical lines are optional in tables. Statements that serve as captions for 
the entire table do not need footnote letters. }\\
\multicolumn{3}{p{251pt}}{$^{\mathrm{a}}$Gaussian units are the same as cg emu for magnetostatics; Mx 
$=$ maxwell, G $=$ gauss, Oe $=$ oersted; Wb $=$ weber, V $=$ volt, s $=$ 
second, T $=$ tesla, m $=$ meter, A $=$ ampere, J $=$ joule, kg $=$ 
kilogram, H $=$ henry.}
\end{tabular}
\label{tab1}
\end{table}


\subsection{Resolution }
The proper resolution of your figures will depend on the type of figure it 
is as defined in the ``Types of Figures'' section. Author photographs, 
color, and grayscale figures should be at least 300dpi. Line art, including 
tables should be a minimum of 600dpi.

\subsection{Vector Art}
In order to preserve the figures' integrity across multiple computer 
platforms, we accept files in the following formats: .EPS/.PDF/.PS. All 
fonts must be embedded or text converted to outlines in order to achieve the 
best-quality results.


\subsection{Accepted Fonts Within Figures}
When preparing your graphics IEEE suggests that you use of one of the 
following Open Type fonts: Times New Roman, Helvetica, Arial, Cambria, and 
Symbol. If you are supplying EPS, PS, or PDF files all fonts must be 
embedded. Some fonts may only be native to your operating system; without 
the fonts embedded, parts of the graphic may be distorted or missing.

A safe option when finalizing your figures is to strip out the fonts before 
you save the files, creating ``outline'' type. This converts fonts to 
artwork what will appear uniformly on any screen.

\subsection{Using Labels Within Figures}

\subsubsection{Figure Axis labels }
Figure axis labels are often a source of confusion. Use words rather than 
symbols. As an example, write the quantity ``Magnetization,'' or 
``Magnetization M,'' not just ``M.'' Put units in parentheses. Do not label 
axes only with units. As in Fig. 1, for example, write ``Magnetization 
(A/m)'' or ``Magnetization (A$\cdot$m$^{-1}$),'' not just ``A/m.'' Do not label axes with a ratio of quantities and 
units. For example, write ``Temperature (K),'' not ``Temperature/K.'' 

Multipliers can be especially confusing. Write ``Magnetization (kA/m)'' or 
``Magnetization (10$^{3}$ A/m).'' Do not write ``Magnetization 
(A/m)$\,\times\,$1000'' because the reader would not know whether the top 
axis label in Fig. 1 meant 16000 A/m or 0.016 A/m. Figure labels should be 
legible, approximately 8 to 10 point type.

\subsubsection{Subfigure Labels in Multipart Figures and Tables}
Multipart figures should be combined and labeled before final submission. 
Labels should appear centered below each subfigure in 8 point Times New 
Roman font in the format of (a) (b) (c). 

\subsection{File Naming}
Figures (line artwork or photographs) should be named starting with the 
first 5 letters of the author's last name. The next characters in the 
filename should be the number that represents the sequential 
location of this image in your article. For example, in author 
``Anderson's'' paper, the first three figures would be named ander1.tif, 
ander2.tif, and ander3.ps.

Tables should contain only the body of the table (not the caption) and 
should be named similarly to figures, except that `.t' is inserted 
in-between the author's name and the table number. For example, author 
Anderson's first three tables would be named ander.t1.tif, ander.t2.ps, 
ander.t3.eps.

\subsection{Referencing a Figure or Table Within Your Paper}
When referencing your figures and tables within your paper, use the 
abbreviation ``Fig.'' even at the beginning of a sentence. Do not abbreviate 
``Table.'' Tables should be numbered with Roman Numerals.

\subsection{Checking Your Figures: The IEEE Graphics Analyzer}
The IEEE Graphics Analyzer enables authors to pre-screen their graphics for 
compliance with IEEE Transactions and Journals standards before submission. 
The online tool, located at
\underline{http://graphicsqc.ieee.org/}, allows authors to 
upload their graphics in order to check that each file is the correct file 
format, resolution, size and colorspace; that no fonts are missing or 
corrupt; that figures are not compiled in layers or have transparency, and 
that they are named according to the IEEE Transactions and Journals naming 
convention. At the end of this automated process, authors are provided with 
a detailed report on each graphic within the web applet, as well as by 
email.

For more information on using the Graphics Analyzer or any other graphics 
related topic, contact the IEEE Graphics Help Desk by e-mail at 
graphics@ieee.org.

\subsection{Submitting Your Graphics}
Because IEEE will do the final formatting of your paper,
you do not need to position figures and tables at the top and bottom of each 
column. In fact, all figures, figure captions, and tables can be placed at 
the end of your paper. In addition to, or even in lieu of submitting figures 
within your final manuscript, figures should be submitted individually, 
separate from the manuscript in one of the file formats listed above in 
Section \ref{formats}. Place figure captions below the figures; place table titles 
above the tables. Please do not include captions as part of the figures, or 
put them in ``text boxes'' linked to the figures. Also, do not place borders 
around the outside of your figures.

\subsection{Color Processing/Printing in IEEE Journals}
All IEEE Transactions, Journals, and Letters allow an author to publish 
color figures on IEEE Xplore\textregistered\ at no charge, and automatically 
convert them to grayscale for print versions. In most journals, figures and 
tables may alternatively be printed in color if an author chooses to do so. 
Please note that this service comes at an extra expense to the author. If 
you intend to have print color graphics, include a note with your final 
paper indicating which figures or tables you would like to be handled that 
way, and stating that you are willing to pay the additional fee.


\section{Conclusion}

A conclusion section is not required. Although a conclusion may review the main points of the paper, do not replicate the abstract as the conclusion. A conclusion might elaborate on the importance of the work or suggest applications and extensions. 

\section*{Acknowledgment}

The preferred spelling of the word ``acknowledgment'' in American English is without an ``e'' after the ``g.'' Use the singular heading even if you have many acknowledgments. Avoid expressions such as ``One of us (S.B.A.) would like to thank . . . .'' Instead, write “F. A. Author thanks ... .” In most cases, sponsor and financial support acknowledgments are placed in the unnumbered footnote on the first page, not here.

\section*{References and Footnotes}

\subsection{References}

References need not be cited in text. When they are, they appear on the line, in square brackets, inside the punctuation.  Multiple references are each numbered with separate brackets. When citing a section in a book, please give the relevant page numbers. In text, refer simply to the reference number. Do not use ``Ref.'' or ``reference'' except at the beginning of a sentence: ``Reference [3] shows . . . .'' Please do not use automatic endnotes in {\em Word}, rather, type the reference list at the end of the paper using the ``References'' style.

Reference numbers are set flush left and form a column of their own, hanging out beyond the body of the reference. The reference numbers are on the line, enclosed in square brackets. In all references, the given name of the author or editor is abbreviated to the initial only and precedes the last name. Use them all; use {\em et al.} only if names are not given. Use commas around Jr., Sr., and III in names. Abbreviate conference titles.  When citing IEEE transactions, provide the issue number, page range, volume number, year, and/or month if available. When referencing a patent, provide the day and the month of issue, or application. References may not include all information; please obtain and include relevant information. Do not combine references. There must be only one reference with each number. If there is a URL included with the print reference, it can be included at the end of the reference.

Other than books, capitalize only the first word in a paper title, except for proper nouns and element symbols. For papers published in translation journals, please give the English citation first, followed by the original foreign-language citation. See the end of this document for formats and examples of common references. For a complete discussion of references and their formats, see the IEEE style manual at www.ieee.org/authortools.

\subsection{Footnotes}

Number footnotes separately in superscripts (Insert $\mid$ Footnote).\footnote{It is recommended that footnotes be avoided (except for the unnumbered footnote with the receipt date on the first page). Instead, try to integrate the footnote information into the text.}  Place the actual footnote at the bottom of the column in which it is cited; do not put footnotes in the reference list (endnotes). Use letters for table footnotes (see Table I). 










\bibliography{references}{}
\bibliographystyle{IEEEtran}
\end{document}
